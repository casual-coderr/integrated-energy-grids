\documentclass[10pt]{article}
\usepackage{graphicx} % Required for inserting images
\usepackage{url}
\usepackage{hyperref}
\title{IEG_Problems_Lecture1}
\author{martavictoriaperez }
\date{January 2025}

\usepackage[margin=1in]{geometry} 
\usepackage{amsmath,amsthm,amssymb, graphicx, multicol, array}
 
\newcommand{\N}{\mathbb{N}}
\newcommand{\Z}{\mathbb{Z}}
 
\newenvironment{problem}[2][Problem]{\begin{trivlist}
\item[\hskip \labelsep {\bfseries #1}\hskip \labelsep {\bfseries #2.}]}{\end{trivlist}}

\begin{document}
 
\title{\textbf{Lecture 4: DC Optimal Power Flow}}
\author{
%Your name\\
DTU Course 46770: Integrated Energy Grids }
\maketitle
\begin{problem}{4.1}

This is a continuation of Problem 3.2 from Lecture 3.

Let us assume now that we are in an hour with an excess of wind generation in Denmark and a deficit in other countries so that the power injection $p_i$ of the different countries is as follows:

\

Germany= - 2 MW, DK1=5 MW, DK2=6 MW, Norway = - 8 MW, Sweden = -1 MW, 

\

Determine the voltage angles $\theta_i$  and the flows $p_l$ in the lines of the network. Assume that $\theta_0$=0; i.e. the reference bus is at node 0 (Germany); and the reactance in all links is $x_l$=1.

\

\textit{Hint 1: Remember the relations between power injection $p_i$ in every node and power flows $p_l$ presented in Lecture 4.}

$p_i=\sum_j {L_{ij} \theta_j } $

\

$p_l=\frac{1}{x_l}  \sum_j{K_{lj} \theta_j}$       

\

$L_{ij}=\sum_l{K_{il} \frac{1}{x_l} K_{lj}}$

\

Hint 2: you can use \href{https://numpy.org/doc/2.1/reference/generated/numpy.linalg.solve.html}{numpy.linalg.solve} to solve the linear equation system. 



\end{problem}

\

\begin{problem}{4.2}

This is a continuation of Problem 3.2 from Lecture 3.

\begin{itemize}

\item[a)] Assuming that the reactance in the links is $x_l$=1, calculate the Power Transfer Distribution Factor (PTDF) matrix.

\item[b)] Assuming the power injection pattern described in Problem 4.1 determine the flows in the lines of the network. 

\end{itemize}

\end{problem}

\


\begin{problem}{4.3}
This is a continuation of Problem 3.3 from Lecture 3.
Using the Python package \href{https://networkx.org/}{networkX}.

\begin{itemize}
\item[a)] Assuming that the reactance in the links is $x_l$=1, calculate the Power Transfer Distribution Factor (PTDF) matrix.

\item[b)] Assuming the power injection pattern described in Problem 4.1 determine the power flows in the lines of the network and plot them.

\item[c)] Assume now that the links unitary reactance is $x_{l}$=[1, 0.5, 0.5, 0.5, 1], calculate the weighted Laplacian (or susceptance matrix),  the PTDF matrix, the power flows and plot them. 

\end{itemize}

\end{problem}

\

\begin{problem}{4.4}
This is a continuation of Problem 3.4 from Lecture 3. For the synchronous zone corresponding to Scandinavia:

\begin{itemize}
\item[a)] Add to the network object the information on the links susceptance, calculate the weighted Laplacian (or susceptance matrix), and the Power Transfer Distribution Factor (PTDF) matrix.

\item[b)] Assuming that power injection in the nodes increases linearly from -1 in the first node to +1 in the last node, calculate and plot the power flows in the network.

\end{itemize}

\end{problem}

%\begin{proof}[Solution]
%Write a solution here
%\end{proof}

\end{document}


 

